% Adapted from layout by Jacob Vanderplas and Gael Varoquaux
%  http://gael-varoquaux.info
%

\documentclass[12pt]{article}
\usepackage{fontspec}
\usepackage{titlesec}
\usepackage[letterpaper,
	    top=1.5cm,
	    left=1.35cm,
	    width=18.2cm,
	    bottom=2cm
	    ]{geometry}
\usepackage{color,hyperref}
\usepackage{amsfonts}
\usepackage{enumitem}
\usepackage{calc}
\usepackage{multirow,bigdelim}
\usepackage{changepage}
\usepackage{graphicx}
\usepackage{fancyhdr}

\definecolor{deepblue}{rgb}{0,.2,.5}
\definecolor{darkblue}{rgb}{0,.1,.3}
\definecolor{deeppurple}{rgb}{.36,.11,.56}
\definecolor{myblue}{rgb}{.01,0.21,0.71}
\definecolor{mypurple}{rgb}{.44,0.27,0.51}
\definecolor{gray}{rgb}{.5, .5, .5}
\definecolor{darkyellow}{rgb}{.46, .49, 0}

\hypersetup{pdftex,  % needed for pdflatex
  breaklinks=true,  % so long urls are correctly broken across lines
  colorlinks=true,
  urlcolor=mypurple,
  %linkcolor=darkblue,
  %citecolor=darkgreen,
  }

\usepackage{ifxetex}

\ifxetex
  \usepackage{fontspec}

  %\fontspec{Fontin}[
  %	  SmallCapsFont={}
  %]

  \defaultfontfeatures{Ligatures=TeX} % To support LaTeX quoting style
  %\setmainfont{fonts/Fontin-Sans-R-45b.otf}[
%	BoldFont = Fontin-Sans-B-45b.otf,
%	ItalicFont = Fontin-Sans-I-45b.otf,
%	BoldItalicFont = Fontin-Sans-BI-45b.otf,
%	SmallCapsFont = Fontin-Sans-SC-45b.otf
 % ]
  %\setsansfont{Fontin Sans}
  \setromanfont{Linux Libertine}
  \setsansfont{GeosansLight}
\else
  \usepackage[T1]{fontenc}
  \usepackage[utf8]{inputenc}
\fi


\def\mydot{\textcolor{deeppurple}{\rule{1ex}{1ex}}}
\newlength\sidebarwidth
\setlength\sidebarwidth{3cm}
\newcommand{\topic}[3][]%
	 {\pagebreak[2]%
	 \vspace{.2cm}
	 \begin{minipage}{\textwidth}
         \phantomsection\addcontentsline{toc}{section}{#1}%
         \nopagebreak\hspace{0in}%
         \nopagebreak\begin{minipage}[t]{\sidebarwidth - .2cm}
         \raggedleft \bf\sc 
	 \color{deeppurple}{\large #2}
	 \end{minipage}%
	 \hfill
	 \begin{minipage}[t]{\linewidth - \sidebarwidth}
	 \nopagebreak{\color{deeppurple}%
		    \rule{0pt}{\baselineskip}%
		    \rule{\linewidth}{2.5pt}%
		    \llap{\raisebox{.3\baselineskip}{\sf #1}}%
		    \vspace*{.1\baselineskip}%
		    }%
	 #3%
	 \end{minipage}
	 \end{minipage}}

	 \newcommand{\subtopic}[3][]
	 {\begin{minipage}{\textwidth}
	 \vspace*{.4\baselineskip}
         \nopagebreak\hspace{0in}%
         \nopagebreak\begin{minipage}[t]{\sidebarwidth - .2cm}
	 % Super posh: using semi-bold condensed fonts. Works only with
	 % lmodern
         \raggedleft {\sf\fontseries{sbc}\selectfont #2}
	 %{\small\sl\\[-0.2\baselineskip] #1}
         {\\[-0.2\baselineskip] \textcolor{gray}{\footnotesize #1}}
	 \end{minipage}%
	 \hfill
	 \begin{minipage}[t]{\linewidth - \sidebarwidth}
	 #3%
	 \end{minipage}%
	 \vspace*{.2\baselineskip plus 1\baselineskip minus
	 .2\baselineskip}%
	 \end{minipage}}

\newenvironment{mywidth}{\begin{adjustwidth}{\sidebarwidth}{}}{\end{adjustwidth}}

\parskip = 0.1in
\parindent = 0.0in

\titlespacing*{\subsection}{0pt}{.1in}{0in}
\textheight=10in
\topmargin=-1.3in

\pagestyle{fancy}
\renewcommand{\headrulewidth}{0pt}
\rfoot{\footnotesize\it \copyright\,Jason Siefken, 2016 \ \makebox(30,5){\includegraphics[height=1.2em]{by-sa.pdf}}}

\begin{document}
	{
		\begin{minipage}[t]{\textwidth}
			\color{deeppurple}\sc {\Huge Math 281-3}
			\hfill Accelerated Mathematics for ISP: First Year
			\hfill Spring 2016

			\vspace{-8pt}
			\color{deeppurple}{\rule{\columnwidth}{3pt}}

			\vspace{.2cm}
		\end{minipage}
	}
	{
		\begin{minipage}[t]{.5\textwidth}
			\begin{itemize}[leftmargin=2cm, itemsep=0ex, parsep=.5ex, labelindent=-4ex, %
		  label={}]
	  			%\item[{\sf\color{gray}{Course:}}] MATH 281-2 %{\footnotesize [CRN 22031]}
				\item[{\sf\color{gray}{Instructor:}}] Jason Siefken\\ (\url{siefkenj@math.northwestern.edu})
				\item[{\sf\color{gray}{Class:}}] MTWF 11:00--11:50 in ISP 203
				\item[{\sf\color{gray}{Dicussion Section:}}] Th 11:00--11:50 in ISP 203
				\item[{\sf\color{gray}{Webpge:}}] {\footnotesize \url{http://www.math.northwestern.edu/~siefkenj/math281-3}}
			\end{itemize}
		\end{minipage}
		\begin{minipage}[t]{.5\textwidth}
			\begin{itemize}[leftmargin=2.5cm, itemsep=0ex, parsep=.5ex, labelindent=-4ex, %
		  label={}]
				\item[{\sf\color{gray}{Office:}}] Lunt 213
				\item[{\sf\color{gray}{Office Hours:}}] T 1:00--2:00, W 2:00--4:00, or by appointment
				\item[{\sf\color{gray}{Textbook:}}] {\small {\it Introduction to Linear Algebra
					for Science and Engineering} Second Edition, by Norman and Wolczuk}
			\end{itemize}
		\end{minipage}
		\vspace{.1cm}
		\color{deeppurple}{\hrule}
	}

	\vspace{.2cm}
	

	\small
	\hspace{\sidebarwidth}\begin{minipage}[t]{\textwidth - \sidebarwidth}
	{
		\parskip=.1cm
	Math 281-3 finished the first year of our accelerated dive into mathematics by exploring
	Linear Algebra, the theory that has been lurking in the background of much
	of what we've done over the last two terms.
	
	Linear Algebra is the study of vectors,
	``flat spaces'' like lines and planes, and linear transformations like 
	rotations and scalings.  Vectors originated in the study of physics and
	the 3D world, but through the mathematical practice of \emph{abstraction},
	we now use vectors to represent non-spacial things like music in addition
	to computer graphics and physical forces.
	
	Transformations are functions that move vectors around, and in this class
	we will focus on \emph{linear transformations}.  Why?  Because although
	mankind has strived to understand the non-linear phenomena of the universe,
	we haven't gotten very far---the non-linear equations governing fluid flow
	still haven't been solved!  However, we have a complete theory of linear
	equations and linear transformations.  Our approach to answering general questions
	about the universe is often to convert the problem into a linear one---one that we
	can actually understand.
	}
	\end{minipage}
	
	\topic{Learning Outcomes}{
		After taking this course, you will be able to:

		\vspace{-.4cm}
		\begin{itemize}[leftmargin=1cm, itemsep=0ex, parsep=.5ex, labelindent=-4ex, label={\mydot}]
			\item Solve systems of linear equations and matrix equations, write vectors
				in different bases, use the geometry of subspaces
				like row spaces, column spaces, null spaces, and eigen spaces to solve
				problems, and switch between geometric and algebraic points
				of view to aid problem solving.
			\item Work independently to understand concepts and procedures that have
				not been previously explained to you.
			\item Clearly and correctly express the mathematical ideas of linear algebra
				to others.
		\end{itemize}
	}


	\topic{Prerequisites}{
		To be prepared for this course, you should have 
		a solid understanding of vectors in $\mathbb {R}^n$, dot and cross products, and complex
		numbers.
	}
	
	\topic{To Succeed}{~}

		\begin{mywidth}
			Learning is hard! It is exercise for the mind, and like exercise,
			when you're doing it, it feels pretty uncomfortable (and if it
			doesn't, you're probably doing it wrong).  Here are some
			tips to help you succeed academically (getting the grade you want)
			and intellectually (learning the most you can).
			
			\vspace{-.4cm}
			\begin{itemize}[leftmargin=1cm, itemsep=0ex, parsep=.5ex, labelindent=-4ex, label={\mydot}]
				\item Form a regularly-meeting study group of 3--4 people.  Having
					others studying around you will help you study, and having someone
					to talk about confusing problems with will help you both
					productively struggle (struggling with others is how real-world
					problems are solved).

				\item Read the textbook \emph{before} class.  In class we will be
					working on problems that we haven't gone over before.
					If you expose yourself to the concepts 
					prior to class, you'll get a lot more out of it.

				\item Visit the tutors, paid for by the university,
					at the {\bf Calculus, Chemistry and Physics Resource Room} at 
					Tech HG04 and Allison 1021 (\url{http://www.math.northwestern.edu/undergraduate/tutoring-advising/tutoring/}).
					Also, take advantage of the ISP upperclassmen.  They've been through
					this struggle before!
			\end{itemize}
		\end{mywidth}

	\topic{Assessment}{~}

	\subtopic[15\%]{Midterm 1}{In-class midterm on Thursday, April 21.}
	\subtopic[15\%]{Midterm 2}{In-class midterm on Thursday, May 12.}
	\subtopic[30\%]{Homework and Quizzes}{
		{\bf Homework:} Homework will be assigned throughout the term and typically
			due at the beginning of class on Thursdays.  You are encouraged
			to work together to solve homework problems, but \emph{you} must write
			up your solutions to be turned in.
			
			Some homework will focus on explaining problems rather 
			than just ``solving'' them.  If you'd like your write-ups
			to look like a pro's, I suggest you use the \LaTeX{} typesetting software.
			\LaTeX{} is the industry-standard for scientific write-ups
			in math, physics, chemistry, computer science, and engineering.  It has a
			learning curve but is well worth the effort.  See the course webpage
			for details.

			\vspace{.2cm}
			{\bf Quizzes:} Quizzes will take place at the beginning of discussion section on Thursdays
			(though not necessarily every Thursday).  They may be announced or unannounced.
		}
	\subtopic[40\%]{Final}{A comprehensive 2 hour final will be held on Monday, June 6 at 3:00--5:00 PM in
	ISP 203.}
	
	\topic{Policies}{
		I have carefully planned the midterm dates, so please ensure you are available 
		for each midterm. If you miss a midterm for a justified reason (illness, 
		family affliction, or other reason recognized by Northwestern's policies), 
		I can excuse it for you by weighting other tests more heavily. 
		However, \emph{there will be no makeup exams}.

		If you have a disability/health consideration that may require 
		accommodations, please contact the Office of Services for Students with Disabilities 
		and register for AccessibleNU as soon as possible.  All information will remain confidential.
		\mbox{\url{http://www.northwestern.edu/accessiblenu}}

		For the rest of Northwestern's polices, please see
		\url{http://policies.northwestern.edu}
	}


\end{document}

