\documentclass[11pt, oneside]{article}   	
\usepackage{geometry}                		
\geometry{letterpaper}
\usepackage{graphicx}
\usepackage{amssymb}
\usepackage{amsmath}
\usepackage{url}
\usepackage[parfill]{parskip}
\usepackage[hidelinks]{hyperref}
\usepackage{pdfpages}
\usepackage{framed}
\usepackage{xcolor}
%\usepackage{fullpage}

% define some colors
\definecolor{my-yellow}{HTML}{ECD078}
\definecolor{my-orange}{HTML}{D95B43}
\definecolor{my-red}{HTML}{C02942}
\definecolor{my-purple}{HTML}{542437}
\definecolor{my-teal}{HTML}{53777A}
\definecolor{my-teal2}{HTML}{008E9C}

\colorlet{strand}{my-teal2}
\colorlet{threshold}{my-teal2}
\colorlet{objective}{my-purple}
\colorlet{subobjective}{my-orange}
\colorlet{leftbar}{my-yellow}
\colorlet{example}{my-teal!85!white}


%
% LaTeX magic to create a new enumerate environment
%
\usepackage{xparse}
\usepackage[inline]{enumitem}
\usepackage{framed}
\makeatletter
	% custom counters
	\newcounter{objective}
	\def\objectives@strand{}
	\def\objectives@threshold{}
	\newcommand{\thestrand}{\objectives@strand}
	\newcommand{\thethreshold}{\objectives@threshold}

	\newcommand{\formatstranditem}{%
		{\color{strand}{\sffamily\thestrand}}%
	}
	\DeclareDocumentCommand{\strand}{o +m}{%
		\gdef\objectives@strand{#1}%
		\par\vspace{.7cm}
		{\Large\textbf{(\formatstranditem)} \hspace{.3cm}\textbf{#2}}\par
	}
	\newcommand{\formatstranditemTextOnly}{\thestrand.\thethreshold}
	\newcommand{\formatthresholditem}{%
		{\color{threshold}{\sffamily\formatstranditemTextOnly}}
	}
	\DeclareDocumentCommand{\threshold}{o +m}{%
		\gdef\objectives@threshold{#1}%
		\def\@currentlabel{\formatstranditemTextOnly}% make a \label command work after a question
		\par\formatthresholditem \textbf{#2}\par
	}
	\newcommand{\formatthresholditemTextOnly}{\thestrand.\thethreshold.\theobjective}
	\newcommand{\formatobjectivesitem}{%
		{\color{objective}{\sffamily\small\formatthresholditemTextOnly}}
	}
	\DeclareDocumentEnvironment{objectives}{o}{
		\nopagebreak%
		\begin{list}{\formatobjectivesitem}{
			\usecounter{objective}
			\setlength{\leftmargin}{1.5cm}
			\setlength{\itemsep}{0cm}
			\setlength{\topsep}{0cm}
			\renewcommand\p@enumi{\formatthresholditemTextOnly\ } % this does not do what I want...
			\renewcommand\p@enumii{\formatthresholditemTextOnly\ }
		}
		\advance\@enumdepth\@ne
	}{
		\end{list}%
	}

	\DeclareDocumentEnvironment{subobjectives}{}{%
		\begin{enumerate*}[label={\color{subobjective}{\sffamily(\alph*)}}]
	}{
		\end{enumerate*}
	}
	\DeclareDocumentCommand{\example}{+m}{%
		{\color{example}{(e.g\mbox{.} {#1})}}%
	}
	\DeclareDocumentCommand{\indentpar}{+m}{%
		\hspace{1cm}\begin{minipage}{\dimexpr\textwidth-1cm}%
			#1%
		\end{minipage}%
	}
	\renewenvironment{leftbar}{%
		\def\FrameCommand{{\color{leftbar}{\vrule width 3pt \hspace{10pt}}}}%
		\MakeFramed {\advance\hsize-\width \FrameRestore}%
	}{\endMakeFramed}

\makeatother
%
%
%


\newcommand{\red}[1] {{\color{red} #1}}



						
\begin{document}

\section*{\LARGE Learning Objectives}

These learning objectives are organized in the following hierarchy.

\vspace{.2cm}
\begin{leftbar}
\indentpar{
	\vspace{-.7cm}
	\strand[S]{Strand} This is a broad categorization, and all
	strands are interconnected by providing necessary technical tools or
	motivation to others.

	\indentpar{
		\vspace{.2cm}
		\threshold[Thresh]{Threshold Concept} This is
			a more focused categorization of objectives. In the interest
			of time, specific learning objectives may be skipped, but all
			threshold concepts must be covered to create a comprehensive
			course.
			\vspace{.1cm}
			\begin{objectives}
				\item Learning
				Objective. These are specific topics, often presented
				with examples, and should be used to construct parts of
				lectures and individual exam questions.
			\end{objectives}
	}
}
\end{leftbar}
\vspace{.2cm}

Some general description\ldots

\vspace{.25in}
		

\strand[E]{Example Objectives}
	\threshold[Try]{Try Out Some Thresholds}
	\begin{objectives}
		\item 
		\begin{subobjectives}
			\item Sub-objectives.
			\item And other sub-objectives.
			\item And yet more.
		\end{subobjectives}
		\item
		\begin{subobjectives}
			\item Another one.
			\item And yet more.
		\end{subobjectives}
	\end{objectives}
	


\end{document}  
